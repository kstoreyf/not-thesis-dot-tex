Galaxies form in dense regions of the Universe, and co-evolve with this underlying dark matter structure over cosmic time.
The distribution of galaxies thus traces the large-scale structure of the Universe, encoding cosmological properties such as its mass content and evolution history.
While traditional approaches have proven powerful at extracting this cosmological information from galaxy surveys, there remains much more information and insight to be gained.
Major challenges include the complexities of modeling the connection between galaxies and dark matter at small scales; the dearth of quality data at very large scales; and the extraction of full information from the galaxy distribution at all scales.
In this dissertation, I present new tools{\emdash}in the form of analysis methods, modeling approaches, and data products{\emdash}for addressing these pressing limitations of large-scale structure and galaxy formation studies.

I begin by developing a cosmological inference approach based on simulations and probabilistic machine learning, and demonstrate that incorporating statistics beyond the standard set improves cosmological constraining power.
I next introduce a new statistic for quantifying the distribution of galaxies that avoids shortcomings of traditional statistics.
On the simulation side, I address the issue of modeling the connection between dark matter and galaxies with a new approach using geometric quantities informed by the laws of physics.
I turn to observational data to construct an all-sky, immense-volume quasar catalog that comprises an unprecedented sample for cosmological analysis.
Finally, I explore the discovery potential of deep learning with a new approach to detecting anomalous galaxy images.
These contributions give insights into how galaxies form in their dark matter environments, and open new possibilities for measuring the fundamental properties of the Universe from galaxy clustering.
They also demonstrate the importance of incorporating physically informed models into cosmological inference frameworks.
The tools introduced in this dissertation pave the way for maximizing the information we can extract from upcoming large-scale structure surveys.
They further demonstrate that we can learn much more about cosmology and galaxy formation with the observations we currently have.