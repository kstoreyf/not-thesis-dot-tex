Galaxies form in dense regions of dark matter, and co-evolve with this underlying large-scale structure over cosmic time.
The distribution of galaxies thus encodes cosmological properties such as the mass content and evolution history of the universe.
While traditional methods applied to major galaxy surveys have proven powerful at extracting this cosmological information, we are limited by our analysis methods, modeling approaches, and---in some regimes---observational data.
At small scales, the complexities of galaxy formation and its connection to dark matter are difficult to model.
At large scales, galaxy surveys have yet to map entire regions of the observable universe.
In this dissertation, I present new analysis methods, modeling approaches, and data products for cosmological inference from galaxy clustering and modeling the galaxy--halo connection.

I develop a cosmological inference approach based on simulations and probabilistic machine learning, and demonstrate that augmenting standard clustering statistics with environment-dependent statistics improves cosmological constraining power.
I push further beyond traditional statistics with the development of a new statistic for quantifying the distribution of galaxies that avoids typical shortcomings.
On the simulation side, I address the issue of modeling the connection between dark matter and galaxies with a new approach using machine learning informed by the laws of physics.
I turn to observational data to construct an all-sky, immense-volume quasar catalog that comprises an unprecedented sample for cosmological analysis.
Finally, I explore the discovery potential of deep learning with a new approach to detecting anomalous galaxy images.
These contributions give insights into how galaxies form in their dark matter environments, and develop approaches for incorporating this and other key physical information into cosmological inference frameworks.
The tools introduced in this dissertation pave the way for maximizing the information we can extract from upcoming large-scale structure surveys, as well as demonstrate that we can learn much more about cosmology and galaxy formation with the observations we currently have.