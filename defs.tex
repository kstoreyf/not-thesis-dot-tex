% ----------------------------------------
% Packages
% ----------------------------------------

% 
% Place here your \usepackage's. Some recommended packages are already included.
%

% Graphics:
\usepackage[final]{graphicx}
%\usepackage{graphicx} % use this line instead of the above to suppress graphics in draft copies
%\usepackage{graphpap} % \defines the \graphpaper command

% Uncomment this to indent first line of each section:
% \usepackage{indentfirst}

% Good AMS stuff:
\usepackage{amsthm} % facilities for theorem-like environments
%\usepackage[tbtags]{amsmath} % a lot of good stuff!
\usepackage{amsmath} % a lot of good stuff!
\usepackage{mathtools}


% Fonts and symbols:
\usepackage{amsfonts}
\usepackage{amssymb}

% Set the fonts
\RequirePackage[T1]{fontenc}
\ifxetex
  \RequirePackage[tt=false]{libertine}
\else
  \RequirePackage[tt=false, type1=true]{libertine}
\fi
\RequirePackage[varqu]{zi4}
\RequirePackage[libertine]{newtxmath}


% For typesetting inference rules
\usepackage{mathpartir}
% \usepackage{pftools}  % A local package
\newcommand{\bmmax}{2}
\usepackage{bm}

% Formatting tools:
%\usepackage{relsize} % relative font size selection, provides commands \textsmalle, \textlarger
\usepackage{xspace} % gentle spacing in macros, such as \newcommand{\acims}{\textsc{acim}s\xspace}


\usepackage[T1]{fontenc} % needed for scaling fancy fonts (?)
\usepackage[utf8]{inputenc} % not sure this is needed

% To stop citations overflowing lines
\usepackage{breakcites}

% For citet command
\usepackage{natbib}
% \setcitestyle{%
%     authoryear,%
%     open={[},close={]},citesep={;},%
%     aysep={},yysep={,},%
%     notesep={, }}
% \let\cite\citep

% for list of fig and table spacing
\usepackage[titles]{tocloft}

%%%% MACROS %%%%%

%%%% general %%%%%

% anomalies uses subcaption and aemulus uses subfig.
% they don't play together. gonna have to update anomalies to subfig, best option i think
%\usepackage{subcaption}
\usepackage[caption=false]{subfig}
\usepackage{bm}		% Bold maths symbols, including upright Greek
\usepackage{verbatim}
\usepackage[export]{adjustbox}
\usepackage{enumerate}
\usepackage[shortlabels]{enumitem}

\usepackage{epigraph}
\setlength{\epigraphrule}{0in}
\setlength{\epigraphwidth}{4in}
\setlength{\afterepigraphskip}{8ex}

\frenchspacing
% not positive want these sloppys, but in quaia paper; TODO look into
\sloppy\sloppypar
% ex is height of x, em is width of M
\newcommand{\emdash}{\rule[0.6ex]{1.2em}{0.12ex}}

% misc
\newcommand{\documentname}{\textsl{Chapter}\xspace}
\newcommand{\eqt}[1]{equation~(\ref{#1})}

% units
\newcommand{\hmpc}{$h^{-1}\,$Mpc}
\newcommand{\hMpc}{h\inv\,\mathrm{Mpc}\xspace}
\newcommand{\hGpc}{h\inv\,\mathrm{Gpc}\xspace}
\newcommand{\Mpch}{h\,\mathrm{Mpc}\inv\xspace}
\newcommand{\Msun}{M_\mathrm{\odot}\xspace}
\newcommand{\hMsun}{h\inv M_\mathrm{\odot}\xspace}

% math
\newcommand{\inv}{^{-1}}
\newcommand{\invp}{^{'-1}}
\newcommand{\T}{^{\mathsf{T}}}
\newcommand{\Tp}{^{'\mathsf{T}}}

% edits
%\newcommand{\new}[1]{\textbf{#1}} % Bolded revisions
\newcommand{\new}[1]{#1} % Un-bold revisions after final acceptance


% comments
\newcommand{\KSF}[1]{\textcolor{teal}{KSF says: #1}}
\newcommand{\ksf}[1]{{\color{teal}KSF says: #1}}
\newcommand{\hogg}[1]{\textcolor{red}{Hogg says: #1}}
%\newcommand{\new}[1]{\textbf{#1}}
%\newcommand{\new}[1]{#1}

% style
\newcommand{\bld}[1]{\bm{#1}}
\usepackage{url}
\urlstyle{tt}


%%%%% Chapter: CFE %%%%%
\newcommand{\rvec}{\bld{r}}

% language
\newcommand{\cf}{2pcf\xspace}
\newcommand{\Est}{The Continuous-Function Estimator\xspace}
\newcommand{\est}{the Continuous-Function Estimator\xspace}
\newcommand{\LS}{LS\xspace}
\newcommand{\foreign}[1]{\textsl{#1}}
\newcommand{\etc}{\foreign{etc}}

% math
\newcommand{\dd}{\mathrm{d}}
\newcommand{\vv}[1]{\bld{v}_\mathrm{#1}}
\newcommand{\TT}[1]{\bld{T}_\mathrm{#1}}
\newcommand{\ff}{\bld{f}}
\newcommand{\NN}[1]{N_\mathrm{#1}}
\newcommand{\GG}[1]{\mathsf{G}_{#1}}


%%%% Chapter: Aemulus %%%%

%%% COMMANDS
\newcommand{\aemulus}{{\sc Aemulus}\xspace}

% symbols
%\newcommand{\inv}{^{-1}}
%\newcommand{\T}{^\top}
\newcommand{\like}{\mathcal{L}}
\newcommand{\cov}[1]{C_\text{#1}}
\newcommand{\covtot}{C_\like}
\newcommand{\simo}{\mathord{\sim}}

% units
% \newcommand{\hMpc}{h\inv\,\mathrm{Mpc}\xspace}
% \newcommand{\hGpc}{h\inv\,\mathrm{Gpc}\xspace}
% \newcommand{\Mpch}{h\,\mathrm{Mpc}\inv\xspace}

% observables
\newcommand{\wprp}{w_\mathrm{p}(r_\mathrm{p})\xspace}
\newcommand{\cfm}{\xi_0(s)\xspace}
\newcommand{\cfq}{\xi_2(s)\xspace}
\newcommand{\upf}{P_\mathrm{U}(s)\xspace}
\newcommand{\mcf}{M(s)\xspace}

% parameters
\newcommand{\om}{\Omega_\mathrm{m}\xspace}
\newcommand{\sig}{\sigma_\mathrm{8}\xspace}
\newcommand{\gf}{\gamma_f\xspace}
\newcommand{\gfs}{\gamma_f f\sigma_8\xspace}
\newcommand{\fsig}{f\sigma_8 \xspace}
\newcommand{\msat}{M_\mathrm{sat}\xspace}
\newcommand{\vbs}{v_\mathrm{bs}\xspace}
\newcommand{\fenv}{f_\mathrm{env}\xspace}





%%%%% Chapter: Eqcosmo %%%%%

% dark properties
\newcommand{\Mtot}{M_\mathrm{tot}\xspace}
\newcommand{\xrms}{x_\mathrm{rms}\xspace}
\newcommand{\vrms}{v_\mathrm{rms}\xspace}
\newcommand{\Mtwoh}{M_\mathrm{200m}\xspace}
\newcommand{\Mfof}{M_\mathrm{200m,FoF}\xspace}
\newcommand{\Rtwoh}{R_\mathrm{200m}\xspace}
\newcommand{\Vtwoh}{V_\mathrm{200m}\xspace}
\newcommand{\Vfof}{V_\mathrm{200m,FoF}\xspace}
\newcommand{\ctwoh}{c_\mathrm{200c}\xspace}

% gal properties
\newcommand{\mstellar}{m_*\xspace}
\newcommand{\rstellar}{r_*\xspace}
\newcommand{\jstellar}{j_*\xspace}
\newcommand{\ssfr}{\mathrm{sSFR}_1\xspace}
\newcommand{\mbh}{m_\mathrm{BH}\xspace}
\newcommand{\mbhpermstellar}{m_\mathrm{BH}/m_*\xspace}
\newcommand{\gminusi}{g-i\xspace}

% other props
\newcommand{\N}[1]{N_\mathrm{#1}}

% features
\newcommand{\x}{\bm{x}\xspace}
\newcommand{\vel}{\bm{v}\xspace}
\newcommand{\Cxx}{C^{(\x\x)}\xspace}
\newcommand{\Cvv}{C^{(\vel\vel)}\xspace}
\newcommand{\CxvS}{C^{(\x\vel),S}\xspace}
\newcommand{\CxvA}{C^{(\x\vel),A}\xspace}



% words
\newcommand{\dark}{\emph{Dark}\xspace}
\newcommand{\hydro}{\emph{Hydro}\xspace}
\newcommand{\field}[1]{\texttt{#1}}


%%%%% Chapter: Quaia %%%%%

\usepackage{makecell}
\renewcommand{\cellalign}{tl}
\renewcommand{\theadalign}{tl}
\renewcommand\cellgape{\Gape[0.1em]}
\renewcommand\theadgape{\Gape[0.1em]}
\usepackage{datatool}
\DTLsetseparator{ = } %spaces on side matter!

\DTLloaddb[noheader, keys={thekey,thevalue}]{quantities}{data/data_quaia/quantities.txt}
\DTLnewdbonloadfalse % allows appending data
\DTLloaddb[noheader, keys={thekey,thevalue}]{quantities}{data/data_quaia/quantities_comparison.txt}

\newcommand{\val}[1]{% 
    % \DTLgetvalueforkey{<cmd>}{<key>}{<db name>}{<ref key>}{<value>}
    % This (globally) sets cmd (a control sequence) to the value of the key specified by <key> (note that this <key> is our value!)
    % in the first row of the database called <db name> which contains the key <ref key> which has the value <value>.
    \DTLgetvalueforkey{\scratchmacro}{thevalue}{quantities}{thekey}{#1}%
    % This checks if hcmdi is null where hcmdi is a control sequence, if it is, then htrue parti is done, otherwise hfalse parti is done.
    % end with xspace
    \DTLifnull{\scratchmacro}{UUU}{\scratchmacro}\xspace
}

% mathy labels 
\newcommand{\Dz}{\Delta z}
\newcommand{\dz}{\Dz/(1+z)}
\newcommand{\zgaia}{z_\mathrm{\emph{Gaia}}}
\newcommand{\zquaia}{z_\mathrm{Quaia}}
\newcommand{\zknn}{z_\mathrm{\emph{k}NN}}
\newcommand{\zsdss}{z_\mathrm{SDSS}}

% abbreviations / formattings
\newcommand{\knn}{$k$NN\xspace}
\newcommand{\Gaia}{\textsl{Gaia}\xspace}
\newcommand{\Gaiapurer}{\Gaia DR3 {`Purer'}\xspace}
\newcommand{\unWISE}{\textsl{unWISE}\xspace}
\newcommand{\SDSS}{\textsl{SDSS}\xspace}
\newcommand{\Catalog}{\Gaia--\unWISE Quasar Catalog\xspace}
\newcommand{\catalog}{\Catalog}
\newcommand{\cat}{Quaia\xspace}

% units
\newcommand{\Gpch}{h^{-1}\,\text{Gpc}}

% latex things
\newcommand\minput[1]{%
  \input{#1}%
  \ifhmode\ifnum\lastnodetype=11 \unskip\fi\fi}

% Shorten often-used values
\newcommand{\Ghi}{\val{Ghi}}
\newcommand{\Glo}{\val{Glo}}
\newcommand{\Gmax}{20.6}
\newcommand{\colorcutstr}{\val{color_cut_str}}



%%%%% Chapter: Anomalies %%%%%

% math and notation
\DeclareMathOperator*{\maxi}{max}
\DeclareMathOperator*{\mini}{min}
\newcommand{\s}[1]{s_\mathrm{#1}}

