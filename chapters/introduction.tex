\graphicspath{{figures/figures_intro/}}


\section{The Cosmological Model}

The universe is impressively well-described by a relatively simple cosmological model.
This model, known as $\Lambda$CDM, posits that the vast majority of the matter in the universe is composed of cold dark matter (CDM), matter whose main interaction is gravitational, and that the accelerating expansion of the universe can be described by a cosmological constant $\Lambda$.
This universe can be parameterized by just a handful of quantities: ... 

In the standard picture of cosmology, the universe began with the Big Bang, quickly followed by a period of inflation which expanded the universe to 
The evolution of matter under gravity causes overdensities 


There is substantial evidence for the $\Lambda$CDM model.


While $\Lambda$CDM is consistent with many of our observations, there remain open questions and inconsistencies that hint at a more complex model or a deeper underlying explanation.




\section{The Connection between Dark Matter and Galaxies}

As dark matter is invisible{\emdash}and has remained undetectable{\emdash}we infer its distribution through luminous tracers.
Galaxies are key tracers: baryonic matter collects in dense regions of dark matter and, over cosmic time, evolves into the galaxies we observe today.

To use the distribution of galaxies to infer the underlying large-scale structure, we must model the connection between galaxies and dark matter, which is related to the intricacies of galaxy formation.

A description that has proven broadly successful is the \emph{halo model}, which ...

The standard halo model only considers the halo mass as the determining factor in the occupation of galaxies.
Analyses have shown that it is critical to take into account the dependency of galaxy occupation on other, secondary halo properties, a phenomenon known as \emph{assembly bias}.



\section{Redshift Surveys}

\begin{figure*}[htp!]
    \centering
    \includegraphics[width=0.6\textwidth]{cfa_survey}
    \caption{A sample of galaxies from the CfA redshift survey. Galaxies in the North Galactic Cap with $M<-18.5$ and velocity $3000<v<6000$ km s$\inv$ are shown in an equal-area sky projection. Filamentary structures, clusters, and voids can be seen visually. \emph{Figure 2b. from \cite{davis_survey_1982}.}}
    \label{fig:cfa_survey}
    \end{figure*}

Over past several decades, we have built up immense 3-dimensional maps of the universe through observations of millions of galaxies.
These involve initial targeting using wide-field imaging, followed by spectroscopic observations to measure galaxy \emph{redshifts}.
The redshifts, measured by the shift of spectral lines from their rest-frame wavelength, indicate the velocity at which a galaxy is moving away from us and thus (assuming the velocity is dominated by the Hubble flow) its distance.
Early photographic observations of galaxies and  analyses of their angular distribution detected non-uniformity below a certain scale \citep{shapley_survey_1932,hubble_distribution_1934,seldner_new_1977,peebles_galaxy_2001}.
These results were extended to three dimensions by early redshift surveys, including \cite{gregory_comaa1367_1978} (238 galaxies around the Coma cluster), the KOS survey (\citealt{kirshner_million_1981}, 133 galaxies), and the CfA survey (\citealt{davis_survey_1982}, $\simo$2400 galaxies).
These confirmed the clustered distribution of galaxies, which can be identified visually even from these small samples. 
Figure~\ref{fig:cfa_survey} shows a sample of galaxies from the CfA survey, about which the authors write that the regions ``exhibit clustering in frothy, almost filamentary, patterns of connectedness surrounding empty holes on the sky.''

Modern redshift surveys have now observed orders of magnitude more galaxies and used their distributions to obtain some of the most stringent constraints on cosmological parameters.
The 2dF Galaxy Redshift survey \citep{Colless2001} observed $\simo$250,000, add the Sloan Digital Sky Survey (SDSS, \citealt{York2000}) observed nearly one million galaxy redshifts.
Galaxy power spectrum analyses of these surveys achieved measurements of the matter density ${\om}$ with a precision of $\simo$10\% \citep{cole_2df_2005, tegmark_cosmological_2006}.
Combining the results of baryon acoustic oscillation and redshift-space distortion analyses, SDSS has measured the parameter to $\sig$ to $\simo$3.5\%; a joint analysis with other cosmological measurements, including the cosmic microwave background, supernovae, and weak lensing, results in $simo$1\% precision.

While galaxies are the dominant tracer observed for large-scale structure surveys given their multitude, other types of sources also provide valuable information.
Quasars, which are extremely luminous emission from active galactic nuclei{\emdash}accreting supermassive black holes{\emdash}at the centers of galaxies, are even more highly biased tracers of matter than galaxies.
Redshift surveys of quasars are the largest-volume maps of the universe we have, as we can observe quasars at immense distances thanks to their high luminosity.
Millions of quasars have been observed, with hundreds of thousands having spectroscopic redshifts.



\section{Cosmological Inference}

The distribution of galaxies is extremely high-dimensional and complex, such that the actual positions of individual galaxies are impossible to model directly.
Luckily, these are largely irrelevant to the underlying cosmology that we care about; rather, the cosmological model makes strong predictions about the \emph{statistical} properties of the large-scale structure.
By computing the statistics of the galaxy distribution, we can compare our observations to models and infer the cosmological model parameters, as well as learn where our models are insufficient to describe the data.

Standard cosmological inference from galaxy clustering relies on low-order statistics of the galaxy distribution. 
The most important statistic is the \emph{two-point correlation function} (\cf), denoted $\xi(r)$, which is defined as the excess probability $\delta P$ of finding a galaxy in a volume element $\delta V$ a given separation $r$ apart from another galaxy compared to a random distribution:
\begin{equation}
    \delta P = n[1+\xi(r)]\delta V ~,
\end{equation}
where $n$ is the mean galaxy number density.
Below scales of $r \sim 10 \hMpc$, the \cf roughly follows a power law, $\xi(r) \sim (r/r_0)^p$, where $r_0$ is a characteristic scale (found to be $r_0 \sim 5 \hMpc$) and $p$ is the power law index, measured to be $p \sim 1.8$.
The \cf can be estimated from galaxy samples with estimators that take into account the finite, irregular window function of the survey, using estimators such as that proposed by \citep{DavisPeebles1983} or \citep{LandySzalay1993}; these are discussed in much more detail in Chapter~\ref{chp:cfe}.
Two-point clustering can also be described in Fourier space by the power spectrum $P(k)$, the Fourier transform of $\xi(r)$, though here we focus on real-space analyses.

We can compare the measured \cf to a model prediction to find the model parameters that most closely describe the data.




\section{Chapter Notes}

In this thesis, I present new tools for studying large-scale structure and galaxy formation from multiple angles.
These tools range from statistical and machine learning methods to simulation-based approaches and observational data catalogs.

In Chapter~\ref{chp:aemulus}, I present an approach for cosmological inference using probabilistic machine learning and N-body simulations, and demonstrate the importance of environment-dependent clustering statistics.
In Chapter~\ref{chp:cfe}, I develop a new statistic for galaxy clustering that obviates issues of traditional, binned clustering statistics.
I explore a new approach to characterizing the dark matter distribution in cosmological simulations using symmetry-preserving quantities in Chapter~\ref{chp:eqcosmo}, relevant to understanding the galaxy--halo connection.
Chapter~\ref{chp:quaia} presents the largest-volume spectroscopic quasar catalog ever constructed, based on \emph{Gaia} and unWISE, which is an unprecedented sample for large-scale structure analyses.
Finally, in Chapter~\ref{chp:anomalies} I look at using deep learning to identify anomalous galaxy images with the goal of widening the discovery possibilities of galaxy surveys.

Chapters~\ref{chp:cfe} and \ref{chp:anomalies} have been refereed and published in the astronomical literature (\emph{The Astrophysical Journal} and \emph{Monthly Notices of the Royal Astronomical Society}, respectively).
Chapter~\ref{chp:aemulus} has been submitted to \emph{The Astrophysical Journal} and is under review.
Chapters~\ref{chp:eqcosmo} and \ref{chp:quaia} are in the final draft stage and will be submitted to journals soon.
The code associated with all of these chapters is publicly available online.\footnote{\url{https://github.com/kstoreyf}}

While the analyses in each Chapter were conducted in collaboration with my co-authors and the support of many others, the majority of the work and the (near-)entirety of the writing in this dissertation is mine. 
I describe the contributions of myself and my coauthors to each Chapter here:
\begin{enumerate}[leftmargin=4\parindent]
    \item[Chapter~\ref{chp:aemulus}:] I developed the idea for this project with Jeremy Tinker and the rest of the \aemulus collaboration. I led the development of this project in conjunction with Jeremy Tinker, with input from Zhongxu Zhai, Joseph DeRose, Risa H. Wechsler, and Arka Banerjee. I implemented all of the code, which used Zhongxu Zhai's code as a starting point and simulations led by Joseph DeRose. I wrote the paper text and received feedback from the rest of the collaboration.
    \item[Chapter~\ref{chp:cfe}:] I developed the idea for this project with David W. Hogg. I implemented the code and worked out the mathematical proof. I wrote the paper with input from David W. Hogg.
    \item[Chapter~\ref{chp:eqcosmo}:] I conceived the idea for this project, and developed it in collaboration with David W. Hogg, Shy Genel, and Soledad Villar. I implemented it with additional feedback from Soichiro Hattori, Austen Gabrielpillai, and Yongseok Jo. I wrote the chapter text, with feedback from these collaborators.
    \item[Chapter~\ref{chp:quaia}:] I developed the idea for this project in collaboration with David W. Hogg and Hans-Walter Rix. I implemented it with additional feedback from Anna-Christina Eilers and Giulio Fabbian. I wrote the text of the chapter with input from these collaborators.
    \item[Chapter~\ref{chp:anomalies}:] The idea for this project was conceived at the Kavli Summer Program in Astrophysics in 2019, in collaboration with Marc Huertas-Company and Alexie Leauthaud. I led project development and implementation, with feedback from Nesar Ramachandra, Francois Lanusse, and J. Xavier Prochaska. Yifei Luo took observations and reduced the data, Song Huang contributed data support, and J. Xavier Prochaska contributed analysis support. I wrote the text of the paper with input from these collaborators.
\end{enumerate}


