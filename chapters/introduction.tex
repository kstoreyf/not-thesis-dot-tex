\section{Chapter Abstract}


\section{The Cosmological Model}


\section{Redshift Surveys}


\section{The Galaxy--Halo Connection}


\section{Cosmological Inference}


\section{Chapter Notes}

In this thesis, I present new tools for studying large-scale structure and galaxy formation from multiple angles.
These tools range from statistical and machine learning methods to cosmological simulations and data catalogs.

In Chapter~\ref{chp-aemulus}, I present an approach for cosmological inference using probabilistic machine learning and N-body simulations, and demonstrate the importance of environment-dependent clustering statistics.
In Chapter~\ref{chp-cfe}, I develop a new statistic for galaxy clustering that obviates issues of traditional, binned clustering statistics.
I explore a new approach to characterizing the dark matter distribution in cosmological simulations using symmetry-preserving quantities in Chapter~\ref{chp-eqcosmo}, relevant to understanding the galaxy--halo connection.
Chapter~\ref{chp-quaia} presents the largest-volume spectroscopic quasar catalog ever constructed, based on \emph{Gaia} and unWISE, which is an unprecedented sample for large-scale structure analyses.
Finally, in Chapter~\ref{chp-anomalies} I look at using deep learning to identify anomalous galaxy images in order to widen the discovery possibilities of galaxy surveys.

Chapters~\ref{chp-cfe} and \ref{chp-anomalies} have been refereed and published in the astronomical literature (\emph{The Astrophysical Journal} and \emph{Monthly Notices of the Royal Astronomical Society}, respectively).
Chapter~\ref{chp-aemulus} has been submitted to \emph{The Astrophysical Journal} and is under review.
Chapters~\ref{chp-eqcosmo} and \ref{chp-quaia} are in the final draft stage and will be submitted to journals soon.
The code associated with all of these chapters is publicly available online.\footnote{\url{https://github.com/kstoreyf}}

While the analyses in each Chapter were conducted in collaboration with my co-authors and the support of many others, the majority of the work and the (near-)entirety of the writing in this dissertation is mine. 
I describe the contributions of myself and my coauthors to each Chapter here:
\begin{enumerate}[leftmargin=4\parindent]
    \item[Chapter~\ref{chp-aemulus}:] I developed the idea for this project with Jeremy Tinker and the rest of the \aemulus collaboration. I led the development of this project in conjunction with Jeremy Tinker, with input from Zhongxu Zhai, Joseph DeRose, Risa H. Wechsler, and Arka Banerjee. I implemented all of the code, which used Zhongxu Zhai's code as a starting point and simulations led by Joseph DeRose. I wrote the paper text and received feedback from the rest of the collaboration.
    \item[Chapter~\ref{chp-cfe}:] I developed the idea for this project with David W. Hogg. I implemented the code and worked out the mathematical proof. I wrote the paper with input from David W. Hogg.
    \item[Chapter~\ref{chp-eqcosmo}:] I conceived the idea for this project, and developed it in collaboration with David W. Hogg, Shy Genel, and Soledad Villar. I implemented it with additional feedback from Soichiro Hattori, Austen Gabrielpillai, and Yongseok Jo. I wrote the chapter text, with feedback from these collaborators.
    \item[Chapter~\ref{chp-quaia}:] I developed the idea for this project in collaboration with David W. Hogg and Hans-Walter Rix. I implemented it with additional feedback from Anna-Christina Eilers and Giulio Fabbian. I wrote the text of the chapter with input from these collaborators.
    \item[Chapter~\ref{chp-anomalies}:] The idea for this project was conceived at the Kavli Summer Program in Astrophysics in 2019, in collaboration with Marc Huertas-Company and Alexie Leauthaud. I led project development and implementation, with feedback from Nesar Ramachandra, Francois Lanusse, and J. Xavier Prochaska. Yifei Luo took observations and reduced the data, Song Huang contributed data support, and J. Xavier Prochaska contributed analysis support. I wrote the text of the paper with input from these collaborators.
\end{enumerate}


